%!TEX root=/Users/sergej/Documents/Master/Thesis/main.tex 
\section{Ricardian model}
\label{sec:Ricardo}
In this subchapter I describe the major assumptions of the Ricardo model by \parencite{Costinot}, which derives the first estimation equation to obtain the fundamental producitivty. \par In general, the model considers a world economy of  $i = 1, \dots, n$ countries and $k = 1, \dots , K $ industries. Further, there is only one factor of production, which is labour. Labour is perfectly mobile across industries and immobile between countries. $L_i $ denotes the number of workers in each country $i$  and $ w_i $ denotes their wage. \par %technology
The authors model the production technology as follows.  Each industry produces a good with a constant returns to scale technology. Further, each good there are indefinitely many varieties $\omega \in \Omega$.  Furhter, they assume that stochastic productivity differences. The fundamental productivity $z_i^k(\omega)$, which denotes how much of a variety $\omega$ may be produced with one unit of labor,  is for each triple of country, industry and variety $(i,k,\omega)$ a random draw from the Fr{\'e}chet distribution. Therefore \[ F^k_i (z) = exp [-(z/z^{k}_i )^{-\theta} ] \] \par
The transport cost are of the following form. Of every unit of a good, which is shipped form industry $k$ in country $i$ to country $j$ only a fraction $1/d^k_{i,j} \leq 1$ arrives. Further the authors assume that for a third country $l$ importing a good $k$ from country $i$ through another country $j$ is more costly than directly importing it. Thus $d^k_{i,l} \leq d^k_{i,j} d^k_{j,l} $ for any third country $l$. \par
The first assumption It is assumed that the market structure is perfect competition. The consumers therefore pay a price equal to the lowest combination of production and transport cost. \[p^k_j(\omega)=\min_{1\leq i \leq I} {c_{i,j}^k(\omega)} \] with the unit production cost are equal to $c^k_{i,j}=\frac{d^k_{i,j} w_i}{z_i^k}$ . \par
Moreover, the consumer preferences are modeled with a  two-tier utility function. The upper-tier is a Cobb-Douglas function and lower tier CES function. 
The choice of the CES function implies that the consumer show a `love-for-variety' property. The consumer welfare with this utility function increases monotonically with the number of goods  for a given level of expenditures on goods and a given price of a variety   \textcite[p. 118]{helpman}. The assumed structure of consumer preferences implies the following relation for the total expenditure of any country $j$ on a variety $\omega$ of a good $k$. 
\begin{align*}
x^k_{j}(\omega) &=  \left(\frac{ p^k_{j}(\omega) }  {  \left( \sum_{\omega' \in \Omega} {p_j^k (\omega')}^{1-\sigma_j^k } \right)^{1/(1-\sigma_j^k) }}    \right)\, \alpha^k_j \,Y_j \quad \text{where} 0 \leq \alpha_j^k < 1,\sigma_{j}^k<1+\theta \quad \text{and} \, \\
\end{align*}
Therefore the expenditure of a destination on a good depends on the pattern of relative prices and the share of income it spends on the particular good. \par
% Further I define $p^k_{j} =  {  \left( \sum_{\omega' \in \Omega} {p_k^j (\omega')}^{1-\sigma_j^k } \right)}^{1/(1-\sigma_j^k) } $  \par
The assumptions guarantee that bilateral trade satisfies the following condition  \begin{align} x^k_{i,j}= \frac{(c^k_{i,j})^{ -\theta} } {\sum_{i'=1}^{I}(c^k_{i',j})^{-\theta } }  \alpha^k_j Y_j \quad \text{and}\quad Y_j=w_j  L_j \end{align}
Therefore following Lemma holds. \begin{align} \ln \left( \frac{{x}_{i,j}^k {x}^{k'}_{i',j}}{{x}_{i,j}^{k'} {x}^{k}_{i',j}} \right) = \theta \ln \left( \frac{{z}_{i}^k {z}^{k'}_{i'}}{{z}_{i}^{k'} {z}^{k}_{i'}} \right)--\theta \ln \left( \frac{ d_{ij}^k d^{k'}_{i',j}}{d_{i,j}^{k'} {d}^{k}_{i',j}} \right) \end{align}
The first log difference of ${x_{i,j}^k} /x_{i,j}^{k'}$ accounts for differences in wages $w_i$ across exporting countries and incomes $Y_j$ across importing countries. Further, the second log difference $\left({x_{i,j}^k} /x_{i,j}^{k'} \right)  \left({x_{i',j}^k} /x_{i,j}^{k'} \right) $ accounts for differences in the expenditure shares $\alpha^k_j$ across destinations. Therefore the ratio of relative exports of country $i$ and $i'$ to country $j$ in industry $k$ and $k'$ is determined by the relative ratio of productivity and the relative ratio of trade cost. Therefore the model  makes Ricardian prediction at the industry level. 

\section{Empirical predictions}
However the prediction above is based on fundamental productivity differences, which can not be empirically observed. In order to make the model empirically viable, a link between fundamental and observed productivity is necessary. \par \textcite{costinot} showed that from assumption 1 the ratio of observed productivity $\tilde{z}^k_i / \tilde{z}^k_{i'}$ for a country pair $i$ and $i'$ %multiplied by the ratio of openness of country $i$ and $i'$  scaled by the exponent one divided by $\theta$ 
links directly to the ratio of fundamental productivity.%, based on this insight they showed the following theorem. 
\par 
\begin{align} \ln \left( \frac{\tilde{x}_{i,j}^k \tilde{x}^{k'}_{i'j}}{\tilde{x}_{i,j}^{k'} \tilde{x}^{k}_{i'j}} \right) = \theta \ln \left( \frac{\tilde{z}_{i}^k \tilde{z}^{k'}_{i'}}{\tilde{z}_{i}^{k'} \tilde{z}^{k}_{i'}} \right)--\theta \ln \left( \frac{ d_{ij}^k d^{k'}_{i'j}}{d_{i,j}^{k'} {d}^{k}_{i'j}} \right) \end{align}
The relation above links openness corrected exports $\tilde{x}_{i,j}^k$ to the observed productivity and trade cost. The first ratio of  $\frac{\tilde{x}_{i,j}^k} { \tilde{x}_{i', j}^{k'} }$ accounts for income differences  $Y_j$ of the importing countries and wage differences $w_i$ across exporting countries. Further, the second ratio $(\frac{\tilde{x}_{i,j}^k}{\tilde{x}^{k'}_{i'j}})/ \frac{tilde{x}^{k}_{i'j}}{\tilde{x}^{k'}_{i'j}}) $ accounts for differences in the expenditure shares $\alpha^k$ %for different industries among importers $j$. Therefore the authors showed that observed productivity links to openness corrected exports.  
\par 
The productivity term in \cref{eq:1} is net of specific trade barriers $\delta_{i,j}$ between country $i$ and $j$ like distance and net of trade barriers $\delta_j^k$ specific imposed by the importing country$j$ on the $k$ goods \footnote{The latter fixed-effect include as well  trade protection in line with the most-favorite nation (MFN) clause of the WTO \parencite{costinot}. The MFN clause is that a country can not offer less favorable conditions to a party e.g. an investor of an agreement than to any other investor in the same specific matter from a third country  \parencite{oecd-mfn}.}. Further the error term $\epsilon^k_{i,j}$ includes variable trade cost and other components.
\par 
Further, to allow the estimation of the equation above the measure of observed productivity has to be specified. In their model setup
the authors showed that the observed ratio of relative productivity is fully reflected  by the inverse ratio of producer price indices.  This result holds in the Ricardo world, however in the context of international product fragmentation, producer prices as well include foreign contributions.
\par
According to \textcite{Costinot} following econometrically equivalent equation may be estimated instead of the previous equation.
  \begin{align} \label{eq:1} \ln \tilde{x}_{i,j}^k=\delta_{i,j}+\delta_j^k + \theta \ln\tilde{z}_i^k+\epsilon^k_{i,j} \end{align}  
The \cref{eq:1} states that the openness corrected exports  $\tilde{x}_{i,j}^k \equiv x_{i,j}^k -  \tilde{\pi}_{i,i} $ from industry $k$ in exporting country $i$ to importing country $j$ are predicted by the observed productivity $\ln\tilde{z}_i^k$,  exporter-importer fixed-effects $\delta_{i,j}$ and importer-industry fixed-effects $\delta_j^k$. \par \textcite{costinot} interpret the equation as an analogue to a `difference-in-difference' estimation. 
The productivity term in \cref{eq:1} is that the first difference of specific trade barriers $\delta_{i,j}$ between country $i$ and $j$ like distance and second differenced of trade barriers $\delta_j^k$ specific imposed by the importing country$j$ on the $k$ goods \footnote{The latter fixed-effect include as well  trade protection in line with the most-favorite nation (MFN) clause of the WTO \parencite{costinot}. The MFN clause states that a country can not offer less favorable conditions to a party e.g. an investor of an agreement than to any other investor in the same specific matter from a third country  \parencite{oecd-mfn}.}. Further the error term $\epsilon^k_{i,j}$ includes variable trade cost and other unobserved time-varrying components. \par
Further, structural RCA measure is obtained as follows. In the first step I estimate $\theta$ and in a second step I obtain the equation with the full set of fixed effects.
   \begin{align} \label{eq:2}\ln {x}_{i,j}^k=\delta_{i,j}+\delta_j^k + \theta \ln{z}_i^k
+\epsilon^k_{i,j} \\
\ln {x}_{i,j}^k=\delta_{i,j}+\delta_j^k + \delta_i^k + \epsilon^k_{i,j} \end{align}
The $\delta_i^k$ in the second relation captures the effect of  $\theta \ln{z}_i^k$ on the bilateral gross exports. 
  \[ z^k_i=e^{{\delta_i^ k}/{\theta}} \] 
\section{Generalization}
In the following, I discuss the effect of sector-specific use of production factors in the cost function. Moreover, I include capital and intermeidate inputs as production factor and I will argue that the effects of sector-specifc use of production factors are similar to sector-specific international sourcing of inputs. I regard both aspects of production as an effect of IPF. The effects of IPF, as generalizing the cost function, are that the ranking of RCA would not only reflect productivity differences but as well sector specfic factor usage and sourcing patterns.  Therefore, I argue  that differences of the RCA ranking for value-added exports or gross exports would reflect the effect of IPF. \par
In the following I introduce international sourcing and sector-specific production factors based on the cost function in \textcite{Shikher}.  \[ c^k_{i,j}=\frac{d^k_{i,j}}{z_{i}^k Y_j} w^{\alpha^k}_i r^{\beta^k}_i \rho^{1-\alpha^k-\beta^k}_{i} \]. Further, I assume that the industries mix intermediate inputs in fixed proportions. The price of inputs $\rho_i$ is therefore a Cobb-Douglas function of industry prices:
\[\rho_{i}= \prod\limits_{m=1}^{K}  p^{\eta_{i,m}}_{i}  \] where $\eta_{i,m} \geq 0$ is the share of industry $m$ goods in the intermediate inputs of industry $k$, such that $\sum_{m=1}^K \eta_{i,m}=1, \forall i$.  For this more general cost function the bilateral trade flows would know depend on the production factor usage in the industries and the usage of input factor prices. The ranking of countries based on the model specific productivity factor would now be confunded by the effects of different factor endowments.  
\par From this general cost function one can still arrive at the relation in theorem if one assumes that the share of production factor used in the production of a good  is not industry specific. Therefore the cost function simplifies as follows  \[ c^k_{i,j}=\frac{d^k_{i,j}}{z_{i}^k } w^{\alpha}_i r^{\beta}_i \rho^{1-\alpha_{i}-\beta_{i}} \]. where
$\rho_{i}= \prod\limits_{m=1}^{K}  p^{\eta_{i,m}}_{i} $ It is easy to see that the cost function together with eq. (1) and assumption 1 would still imply eq. (2). \par Therefore I can interpret the hypothesis that value-added exports affect the  RCA ranking, as a test whether the factor shares of the inputs and other production factors are sector specific. Further, it is easy to extend the cost function to include international sourcing of inputs by introducing another subscript to denote whether the input production factor is sourced domestically or abroad. Similar to the previous argument about sector specific production factors, the international sourcing pattern would confound the picture of RCA rankings.  
%
%\section{Ricardian model -- version 2}
%\label{sec:Ricardo}
%%The model of \textcite{costinot} generalized the Eaton-Kortum multi-country Ricardo model to a multi-industry setup by moving the heterogeneous stochastic productivity to the level of varieties. The main contribution of  \textcite{costinot} was to derive a consistent theoretical alternative to the measure of revealed comparative advantage from \textcite{Balassa} in a multi-country multi-industry setup with imperfect specialization. \par
%
%In this section I  describe the model of \textcite{costinot}. Further, I discuss a possible generalization of the production structure of this model, which allows to discuss a theoretically framework to study the effect of IPF on the structure of RCA.   \par 
%
%The model of \textcite{costinot} is set up as follows. The world economy consists of  $i = 1, \dots, n$ countries and $k = 1, \dots , K $ industries. Further, labour is the only factor of production, which is perfectly mobile across industries and immobile between countries. $L_i $ denotes the number of workers in each country $i$  and $ w_i $ denotes their wage. \par %technology
%
%The authors model the production technology as follows.  Each industry produces a good with a constant returns to scale technology. Further, each good there are indefinitely many varieties $\omega \in \Omega$. Moreover, $z^k_i(\omega)$ denotes the number of varieties $\omega$ ,which can be produced with one unit of labor in country $i$, industry $k$. This labor productivity is for each country, industry and variety a random draw from the Fr{\'e}chet distribution, which is an extreme value distribution.
%Formally, 
% By the choice of the distribution, two parameters $z^k_{i}$ and $\theta$ summarize the complete production technology differences across countries and industries. \par 
%
%The first term describes the fundamental productivity of industry $k$ in country $i$ due to e.g. climate, infrastructure and institutions, which affect all firms in one country's industry. The variations of $z^k_i$ determine the cross-country differences in relative labor productivity. The second parameter $\theta$ measures intra-industry heterogeneity, which means that it reflects the differences of production know-how across varieties.  Further, $\theta$ is the same for all industries and countries. To highlight the implication of this assumption,
%consider e.g. the productivity in the manufacturing industries 'food and beverages' and 'machinery and equipment'  the assumption implies that  productivity differences in both industries are the same in each country.\par %trade cost % technical assumptions
%
%%Transport Cost
%Additionally, the authors assume transport cost of Samuelson{'}s iceberg form. Therefore of every  unit of a good, which is shipped, a certain share of the value  'melts' away  \parencite[p.78]{combes}. Formally, $1/d^k_{i,j} \leq 1$ units arrive of every shipped unit of the good $k$ from country $i$ in country $j$  %\footnote{ The main motivation of iceberg trade cost is that it facilitates general equilibrium modeling, as it allows to not model the transport industry  \parencite{krugman1998}. Empirically however there is no good reason to   \parencite{krugman1998}.}. 
%\par
%Further the model assumes that the market structure is perfect competition. Therefore the consumer pay the lowest price  for each variety of a good, which is equivalent to the cost of production and the shipping costs. The unit cost of production and delivering are as follows $c^k_{i,j}=\frac{d^k_{i,j} w_i}{z_i^k}$ and the price for consumers in country $j$ .  \par 
%
% Further, the authors assume that no cross-country arbitrage is possible, which means that for a third country $l$, it is more expensive to indirectly import a good from  country $i$ through  country $j$ than to directly import the good.  \par % consumer problem
%The utility function of the representative consumer is a two-tier utility function. The upper-tier utility function is a Cobb-Douglas function.  The lower-tier utility function is a constant elasticity of substitution utility function. Therefore the expenditure of any country $j$ on one variety $\omega$ of a good $k$ are equal to
%
% % \footnote{The constant elasticity assumption implies that opening up to trade has no pro-competitive effects.}, which assumes that all variety are symmetric \textcite[p. 117]{helpman}. 
%The consumer preferences show a 'love-for-variety' property, because the consumer welfare increases monotonically with the number of goods  for a given level of expenditures on goods and a given price of a variety  \textcite[p. 118]{helpman}. The assumption implies that the total expenditure of any country $j$ on a variety of a good depends on the pattern of relative prices and the share of income it spends on the particular good.
%  %preferences
%A final assumption, which closes the model is that  trade is balanced trade across countries. 
%
%%Further to show that their model makes a theoretical prediction of a relation between exports and productivity differences, the authors assume that trade cost between country $i$ industry $k$ and country $j$  can be decomposed into a country pair $\delta_{i,j}$ and destination-industry specific trade cost $\delta^k_j $, therefore $\delta^k_{i,j}=\delta^k_j \delta_{i,j}$. The authors prove that under this assumptions the ranking of fundamental productivity fully determines the ranking of relative exports. 

\endinput