%!TEX root=C:/Users/Sergej/Documents/GitHub/Thesis/main.tex
\chapter{Conclusion}
The objectives of this thesis were two-fold.
First, assess whether the impact of IPF on the production process is such that traditional export measures no longer provide a reliable picture of technological comparative advantage.
A second objective was to analyze the association between relative network centrality and technological comparative advantage.
The hypothesis was motivated by the similarity of the interpretation of network centrality as how important an industry in a country in the export network is regarding \$ to the stochastic interpretation of trade shares resulting from productivity draws. \par
To analyze technological comparative advantage, I used an structural RCA measure based on the methodology of\textcite{costinot}.
The authors developed a theoretically consistent measure in a setup with imperfect specialization, multiple industries, and multiple countries.
 I estimated the measure for both domestic value-added and gross exports and compared the results using simple and rank correlations. \par
I proceeded in two steps to construct the structural RCA.
 In a first step I estimated a regression of the log of bilateral trade flows on the log of observed productivity, the inverse of international prices, an exporter-importer fixed effect and an importer-industry pair fixed effect.
  For this regression, I created a sample combining international relative price data from the GGDC \textcite{Inklaar2012}, R\&D data from the ANBERD OECD database and gross exports and value-added data from the \textcite{tiva2}.
  In the second step, I regressed the log of bilateral trade flows on the full set of export-importer, importer-industry and export-industry fixed effects.
   For the second estimation, I used the complete TiVA sample with 56 countries.  \par
In the first regression, I obtained an estimate for $\theta$, which may be interpreted as the inverse of the productivity dispersion.
 Comparing the point estimates of the productivity dispersion parameter to the results in \textcite{eaton} and in \textcite{costinot},
 I find that my point estimates were at the upper bound of the 95 \% confidence interval of the first and my estimates were significantly higher than the results of the latter.
  Moreover, I found that the estimates of $\theta$ using value-added exports showed increased values.
   This results may be explained by the construction of domestic value-added, which are net of double counting, foreign value-added and domestically absorbed exports.
   The variance of Domestic value-added exports band compared to gross exports.
   If fewer variations of the regressand are explained by the same variations of the observed productivity regressor, the estimated inverse of productivity dispersion $\theta$ should be increased.  \par
 Further, to assess the robustness of the estimates I reestimate the regression in the multiplicative form with PPML methods, as suggested by \textcite{silva}.
  The  estimates of the dispersion parameter showed a statistical not significant decreased estimate compared to the log-linear specification.
  An explanation of the result is that in the levels specification there is more variation of the regressand and, therefore, the estimates decreased.
  The results confirmed that the estimated values are not sensitive to the sample or the estimation technique. \par
Comparing the results of the structural RCA with gross exports and domestic value-added exports, I found that the simple and rank correlation coefficients showed very high coefficients.
The result suggests that the sector-specific input and sourcing patterns are similar do not vary strongly across sectors and, therefore, cleaning gross exports of foreign value-added does not change the ranking significantly.
Moreover, I compared the ranking results for the real estate industry to the results of \textcite{Koopman}.
In contrast to the authors, I find that domestic value-added exports do not change the ranking significantly. \par
 The second objective of this thesis was to analyze the empirical relation between relative network centrality and structural RCA.
 In network propagation of shocks literature, it was shown that for a single layer production network that network centrality is a first-order characteristic of how many actors contribute in \$ terms to the network.
 This interpretation is similar to the stochastic interpretation of trade shares as a result of productivity draws.
   Hence,  I  analyzed the association between relative centrality and structural RCA.
   The results of a correlation analysis showed a stronger rank correlation than simple correlation and hence pointed out that the association is rather monotone than linear between both measures.
   In conclusion, the strong empirical correlation supports the hypothesis. \par
  A direction for future work is to establish a theoretical model that explains the strong association of relative network centrality of an industry in a country in the international production network and structural RCA.
  Moreover, future work may analyze the association between a RCA ranking based on domestic value-added exports and the ranking predicted by the Heckscher-Ohlin model.
\endinput
