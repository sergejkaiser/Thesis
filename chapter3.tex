%!TEX root=/Users/sergej/Documents/Master/Thesis/main.tex 
\chapter{Structural Ricardian comparative advantage for value-added trade}
\label{cha:empirical}
In this chapter I briefly describe the Ricardo model of \textcite{costinot}. The contribution of the model is that it shows that relative cost differences predict the pattern of trade specialization at the country-industry level in a multiple country and multiple industries framework. Further,  of more interest here, is the structural Ricardian comparative advantage measure developed in the model. I use this measure to compare RCA rankings advantage for value-added exports and gross exports.   \par
The structure of the chapter is as follows, first I discuss the indicator of value-added exports. Next, I describe the construction of value-added exports and the sample. Further, I describe the empirical methods used. Then, I turn to the estimation of the productivity dispersion parameter and the exporter-industry fixed effects. In the last step I use correlation analysis to assess the association structural RCA for the different export measures. %The correlation analysis is answering the first objective, whether IPF has an important impacts on RCA.
%I expect that if IPF has a strong impact on technological comparative advantage that the sector-specific sourcing and input structures strongly vary in different sectors.  The result of this would be that the rankings based on gross exports and domestic value-added exports would be significantly different. 
\section{What for value-added exports and which indicator of value added exports ?} \label{sec:vax} 
The literature on value-added exports is motivated by the international production fragmentation (IPF). As a consequence of IPF  intermediate inputs cross country boarders frequently in the production process and therefore goods are increasingly double counted in gross exports. This implies means that gross exports overestimate the domestic value-added content of goods \parencite{johnson}. Another consequence of IPF is that the contribution to the production of goods by different countries in terms of value-added may be included in the gross exports of another country. Hence, gross exports overestimates and does not correctly attributes value-added. As gross exports are not able to answer the question ``who is producing for whom'' \textcite{daudin}, this motivated the development of new measurements. \par    
 % different measures to study IPF, VAX
The pioneer work of \textcite{Hummels} (HYI) defined the statistical indicators of vertical specialization to study IPF.  According to HYI a country may engage in vertical specialization in two ways, either importing intermediate inputs to produce goods for export or by exporting intermediate inputs, which are subsequently used as inputs in the production of exports by other countries.  \textcite{Koopman} criticized this indicator of vertical trade for two reasons. First, it assumes that the amount of imported inputs used in domestic and exports industry is the same and certain types of exports violate this assumption. Second, the concept assumes that imports are completely sourced abroad. The authors argued that this assumption no longer necessarily holds for more than two countries. Therefore, they reasoned that a different indicator value-added exports is better suited to study IPF, as it does not include any double counting and correctly attributes value-added in the production process to . %In the next paragraph I describe the intuition behind the domestic value-added concept. I turn to the work of \textcite{Koopman} and \textcite{johnson},  which both put forth accounting frameworks to decompose gross exports into value-added exports (VAX) and further decompose VAX into different items as domestic value-added exports.  
\par
At the country level the framework of \textcite{Koopman} showed how to decompose gross exports into value-added. However, at the industry level decomposing gross exports into value-added exports leads to two different perspectives, the backward linkages perspective and the forward linkages perspective\textcite{wang2013}.  I will outline the different perspectives in the following and In the appendix I describe the math behind the decomposition for the 2 country and 2 goods case. \par 
Backward linkage value-added exports of an industry include the direct domestic value-added of that industry and other upstream domestic industries in the gross exports of the exporting industry. This perspective is based on the destination country's view. It traces the sources of exports back to a country-sector \textcite{wang2013}. \par The forward linkages perspective  traces the value-added of an industry, which is either directly or indirectly  through other industry is used to satisfy foreign final demand. This perspective is a supply side view. It describes how the value-added produced in one industry is used to satisfy foreign final demand through direct and indirect exports \textcite{wang2013}. Further, this perspective is in line with the factor content view of trade. 
\par
%Why use DVA 
% KWW (2014) p.32 "Because domestic value added or
%GDP in a country?s exports describes the characteristics of a country?s production
%(total domestic factor content in exports), it does not depend on where the exports
%are absorbed. For those applications in which a production-based RCA is the right
%measure, we can use GDP in exports to compute RCA."
%WWZ . "In other words, it decomposes GDP (domestic value-added) by industries according to where (i.e., which
%sector/country) it is used. Such a forward linkage perspective is consistent with the
%literature on factor content of trade.
%The traditional RCA ignores both domestic production sharing and
%international production sharing. To be more specific, first, it ignores the fact that a
%country-sector’s value added may be exported indirectly via the country’s exports in
%other sectors. Indirect exports of a sector’s value added should be included in a
%conceptually correct measure of a country’s sector’s comparative advantage. Second,
%it also ignores the fact that a country-sector’s gross exports partly reflect foreign
%contents (which show up in both FVA and a portion of PDC). A conceptually correct
%measure of comparative advantage needs to exclude foreign-originated value added
%and pure double counted terms in gross exports but include indirect exports of a
%sector’s value added through other sectors of the exporting country."
The two perspectives are useful for different purposes \textcite{wang2013}.  First a backward-linkages based view is useful to understand a country's domestic value added which is embodied in it's exports. In the context of RCA, the domestic value-added in gross exports, is consistent with a production based RCA, since it measures the 'total domestic factor content in exports'  \textcite(p.490){Koopman} note. \par
 Second, \textcite{wang2013} describe that  the forward linkages perspective on value-added exports is helpful to understand how much value-added a given sector contributes to a country's exports.  This indicator correctly attributes how much value-added of an industry is directly  and indirectly through further downstream industries is exported across destinations.  The RCA ranking with this indicator shows how efficiently an industry uses the domestic factors of production  \textcite{bladwin}. % Further, they use this indicator to compute an RCA ranking with the ad-hoc Balassa index of RCA in several industries and compared it with gross exports and found significant differences.. \par
%. Based on this indicator they analyze the evolution of RCA over time for the country pair China and the USA in the sector "Electrical and Optical Equipment". \par 
% I will create rankings based on both indicators. The motivation to create an RCA ranking based on backward linkages is that it implies the view that a country's  RCA in an industry is also based on domestic supply chains. Further creating such a ranking allows to compare the results I obtain to \textcite{Koopman}. On the other hand, the forward linkages view may be valuable as it is close to the factor content view of trade. Creating a ranking would indicate that a industry in one country employs the factors of production more efficiently. 
\par %Why 
To obtain data on value-added exports various sources exist. Previous literature on value added trade mainly focused on the WIOD \textcite{Timmer2012} database. In this thesis I use the value-added export data from the TiVA \textcite{tiva2} database. This choice is motivated at first that the estimation of the structural RCA requires matching different databases, which are based on a industry classification similar to the ISIC classification, which is used in TiVA. Further, from a substantial point TiVA data provides a larger country coverage with a more regional diverse focus. In addition, as a \textcite{johnson} showed only one author has employed the TiVA dat. \par
Further, data however is necessary to be able to estimate the structural RCA indicator. In a first step \textcite{costinot}  used a regression of gross exports on the inverse of producer prices instrumented by R\&D to obtain an estimate of an dispersion parameter. To follow their approach I combine  the (value-added) exports data of \textcite{tiva2} TiVA, with  R \&D expenditure from \textcite{stan2} ANBERD and  international producer price data from the GGDC  \parencite{Inklaar2012}. The sample I obtain by combining this sources is the estimation sample. It includes twenty nine countries and twenty two industries at the ISIC Rev.3 level. At the second stage I require only the value-added exports data of the TiVA. This larger dataset includes the twenty two industries at the ISIC Rev.3 classification and fifty six countries.
\par
Further to create the first sample several data managements steps were necessary. To reconcile the industry coverage of TiVA and the GGDC international price data I aggregated the international price data using the ISIC Rev. 3.1 two digits classification. The data management steps enabled me to extend the sample to include a larger proportion of service sectors. In practice, I aggregated  the prices using a weighted average. I specified as the  weights the relative share of the industries in terms of its sector value-added share with country industry data from the  \textcite{OECDSTAN} STAN database. Further, in case of missing data on value-added output data  I aggregated the prices by a simple average. The resulting cross-section sample spans 22 industries and  29 countries for the year 2005.  Moreover in the appendix in table \ref{tab:sumstat} I report the descriptive statistics of the sample.  \par 
Further, compared to the full country coverage of the TiVA November release I excluded some countries because they had no exports to any destination recorded in at least one sector \footnote{ Island, Costa Rica, Brunei Darussalam}. % For the last two countries, the extensive margin at the industry level, which is the number of destinations of non-zero exports from industry divided by the theoretical maximum was less than fifty percent. 
Further, I excluded Saudi Arabia because its exports mainly consist of oil \footnote{ For 2005, the share of petroleum exports accounts for  90\% of the fob exports. Fob denotes the price of a good at the factory excluding delivery and insurance costs \parencite[p.78]{combes}} \parencite{opec}. Moreover, fifteen  countries did not have records on forward linkages value-added exports and were therefore as well omitted from the sample \footnote{Lithuania Latvia, Malta, Malaysia, Philippines, Romania, Rest of the World, Russia, Singapore, Thailand, Tunisia, Taiwan, Vietnam, South Africa, Costa Rica, Brunai Darussalem, Khambodia, Island}.  Thus, the fixed-effects sample includes 22 industries and 43 countries for the year 2005.  
\endinput
