\begin{table}[H]
\footnotesize
\def\sym#1{\ifmmode^{#1}\else\(^{#1}\)\fi}
\centering \caption{Correlation between structural RCA based on gross exports and domestic valued added exports}
\label{tab:exgr-dva}
\begin{tabular}{l*{3}{S}}
  \toprule
ISIC & \multicolumn{1}{c}{Correlation} & \multicolumn{1}{c}{Rank correlation}   & \multicolumn{1}{c}{Difference}\\ 
  \midrule
15-16 & 0.99 \sym{***}& 0.98\sym{***} & 0.01 \\ 
  17-19 & 0.99 \sym{***}& 0.99 \sym{***}& 0.00\\ 
  20 & 1.00 \sym{***}& 1.00 \sym{***}& 0.00  \\ 
  21-22 & 1.00\sym{***} & 0.99\sym{***} & 0.01 \\ 
  23 & 0.98 \sym{***}& 0.95 \sym{***}& 0.03 \\ 
  24 & 1.00 \sym{***}& 0.99 \sym{***}& 0.01 \\ 
  25 & 1.00 \sym{***}& 0.99 \sym{***}& 0.01\\ 
  26 & 1.00 \sym{***}& 1.00 \sym{***}& 0.00   \\ 
  27-28 & 0.99 \sym{***}& 0.99 \sym{***}& 0.00  \\ 
  29 & 1.00 \sym{***}& 0.99 \sym{***}& 0.01 \\ 
  30-33 & 0.99 \sym{***}& 0.99 \sym{***}& 0.00 \\ 
  34-35 & 1.00 \sym{***}& 0.99 \sym{***}& 0.01 \\ 
  36-37 & 1.00 \sym{***}& 0.99 \sym{***}& 0.01 \\ 
  45 & 1.00 \sym{***} & 0.99 \sym{***}& 0.01\\ 
  50-52 & 1.00 \sym{***}& 1.00\sym{***} & 0.00  \\ 
  55 & 1.00\sym{***} & 0.99 \sym{***}& 0.01\\ 
  60-64 & 1.00 \sym{***}& 0.99 \sym{***}& 0.01 \\ 
  65-67 & 1.00\sym{***} & 1.00 \sym{***}& 0.00   \\ 
  70-74 & 1.00 \sym{***}& 1.00 \sym{***}& 0.00   \\ 
  75-95 & 1.00 \sym{***}& 1.00\sym{***} & 0.00   \\ \midrule
  AVG & 1.00 & 0.99 &    0.01 \\ 
   \bottomrule
   \multicolumn{4}{l}{\footnotesize \sym{*} \(p<0.05\), \sym{**} \(p<0.01\), \sym{***} \(p<0.001\)}\\
   \multicolumn{4}{l}{ISIC Rev. 3.1 Code} \\
   \multicolumn{4}{l}{Structural RCA based on domestic valued added exports} \\
   \multicolumn{4}{l}{Benchmark Industry ISIC Rev. 3 01-05 Agriculture}\\
   \multicolumn{4}{l}{Benchmark Country Rest of the World}\\
\end{tabular}
\end{table}

%Conover W.J. (1980) Practical Non-Parametric Statistics, Kendall tau porbability of concordant and discordant ranks
Further,  I report the Country ranking of RCA for the industry of real estate, renting and business activities relative to the US and the agriculture industry.  Singapore has the highest structural comparative advantage. Ireland has the second highest comparative advantage and  Hong Kong the third highest. The relative percentage difference between the country with the highest comparative advantage Singapore and the country with the lowest comparative advantage (Vietnam) is 37 \%. Further, the differences in structural RCA in Europe are quite large. The relative percentage difference of structural comparative advantage between Ireland and Greece is 34 \%. %Excluding Greece, the relative percentage difference between Ireland and Bulgaria is still 26 \%. 
Moreover the table allows to compare the results directly to \textcite{Koopman}, which also computed a RCA ranking for several countries in this industry.
 Comparing the ranking directly to \textcite[Figure 2, p.491]{Koopman} is difficult due to differences in the country coverage. The sample I report includes 56 country, whereas   \textcite{Koopman} sample includes only 29 countries. Further, their sample consists of several country aggregates like EU, EFTA, EU Accession and Rest of Americans, which are absent in the sample here. \par  rankings show that Singapore and Hong Kong show the highest comparative advantage and Vietnam and Indonesia are in both rankings at the bottom of the table. Further, for several countries I observe large differences to the ranking of \textcite{Koopman}, e.g.  Korea, which is ranked the fourth in this ranking, whereas in \citeauthor{Koopman} it is ranked in the middle. Overall the results I report in \cref{tab:rank} show large differences to the results of \textcite{Koopman}. % elaborate
\begin{table}[H]
\scriptsize
\centering\caption{Structural RCA Ranking - Real estate, renting,
and, business activities}
\label{tab:rank}
\begin{tabular}{llclc} \toprule
   &     \multicolumn{2}{c}{Gross Exports}&                \multicolumn{2}{c}{ Value Added Exports}    \\            
        \multicolumn{1}{c}{Rank} &        \multicolumn{1}{c}{RCA} &          \multicolumn{1}{c}{Country} &           \multicolumn{1}{c}{RCA} &          \multicolumn{1}{c}{Country}    \\    \midrule
1 & IRL & 1.44 & SGP & 1.42 \\ 
  2 & SGP & 1.43 & IRL & 1.41 \\ 
  3 & HKG & 1.38 & HKG & 1.39 \\ 
  4 & SWE & 1.35 & SWE & 1.36 \\ 
  5 & KOR & 1.35 & KOR & 1.36 \\ 
  6 & LUX & 1.33 & CHE & 1.32 \\ 
  7 & CHE & 1.32 & GBR & 1.31 \\ 
  8 & GBR & 1.30 & BEL & 1.28 \\ 
  9 & BEL & 1.27 & LUX & 1.27 \\ 
  10 & AUT & 1.25 & AUT & 1.26 \\ 
  11 & HRV & 1.24 & ISR & 1.26 \\ 
  12 & USA & 1.24 & USA & 1.24 \\ 
  13 & ISR & 1.23 & HRV & 1.24 \\ 
  14 & ITA & 1.22 & ITA & 1.23 \\ 
  15 & RUS & 1.22 & RUS & 1.23 \\ 
  16 & CYP & 1.22 & DEU & 1.22 \\ 
  17 & DEU & 1.21 & CYP & 1.22 \\ 
  18 & SVN & 1.17 & SVN & 1.19 \\ 
  19 & PRT & 1.17 & PRT & 1.18 \\ 
  20 & NLD & 1.17 & FIN & 1.18 \\ 
  21 & IDN & 1.16 & TWN & 1.17 \\ 
  22 & FIN & 1.16 & NLD & 1.17 \\ 
  23 & SVK & 1.16 & FRA & 1.17 \\ 
  24 & TWN & 1.15 & NOR & 1.16 \\ 
  25 & FRA & 1.15 & SVK & 1.16 \\ 
  26 & NOR & 1.15 & IDN & 1.16 \\ 
  27 & ESP & 1.15 & ESP & 1.16 \\ 
  28 & ROU & 1.15 & CHN & 1.15 \\ 
  29 & CHN & 1.14 & HUN & 1.14 \\ 
  30 & HUN & 1.13 & DNK & 1.14 \\ 
  31 & DNK & 1.13 & ROU & 1.14 \\ 
  32 & CZE & 1.12 & CZE & 1.13 \\ 
  33 & POL & 1.12 & POL & 1.13 \\ 
  34 & JPN & 1.12 & JPN & 1.12 \\ 
  35 & LVA & 1.11 & LVA & 1.12 \\ 
  36 & MYS & 1.10 & MYS & 1.10 \\ 
  37 & MEX & 1.10 & MEX & 1.10 \\ 
  38 & THA & 1.09 & THA & 1.10 \\ 
  39 & AUS & 1.09 & AUS & 1.09 \\ 
  40 & TUN & 1.07 & TUN & 1.07 \\ 
  41 & PHL & 1.07 & PHL & 1.07 \\ 
  42 & EST & 1.07 & EST & 1.07 \\ 
  43 & LTU & 1.05 & CAN & 1.06 \\ 
  44 & CAN & 1.04 & LTU & 1.05 \\ 
  45 & NZL & 1.03 & BGR & 1.04 \\ 
  46 & BGR & 1.03 & NZL & 1.04 \\ 
  47 & CHL & 1.02 & CHL & 1.03 \\ 
  48 & ROW & 1.00 & ZAF & 1.01 \\ 
  49 & ZAF & 0.99 & ROW & 1.00 \\ 
  50 & BRA & 0.99 & BRA & 0.99 \\ 
  51 & TUR & 0.97 & TUR & 0.98 \\ 
  52 & GRC & 0.95 & GRC & 0.96 \\ 
  53 & IND & 0.93 & IND & 0.94 \\ 
  54 & ARG & 0.92 & ARG & 0.93 \\ 
  55 & COL & 0.91 & COL & 0.92 \\ 
  56 & VNM & 0.88 & VNM & 0.89  \\
\bottomrule 
\end{tabular}
\end{table}
\begin{comment}
The \cref{tab:prod_1995} shows the Pearson and Spearman correlation of structural RCA indicator measure for the years 1995 and 2005 based on domestic value-added exports. Overall the table shows a very strong correlation between the measures of both Spearman and Pearson correlation. The Pearson correlation spans from 0.76 for the construction sector to 0.95 for machinery. The results for most industries are very similar for both correlation measures. The majority of industries show a slightly stronger Pearson correlation than Spearman correlation. The relation between both variables is thus rather linear. The table suggests that the 'revealed' comparative advantage measure has a stable ranking over time.\\
\begin{table}[H]
\def\sym#1{\ifmmode^{#1}\else\(^{#1}\)\fi}
\footnotesize
\centering \caption{Correlation of structural RCA in 1995 \& 2005 }
\label{tab:prod_1995}
\begin{tabular}{l*{3}{S}}
  \toprule
ISIC & \multicolumn{1}{c}{Correlation} & \multicolumn{1}{c}{Rank correlation}   & \multicolumn{1}{c}{Difference}\\ 
  \midrule
15-16 & 0.87\sym{***} & 0.83 & 0.04 \\ 
  17-19 & 0.94 \sym{***}& 0.94 & 0.00  \\ 
  20 & 0.90 \sym{***}& 0.90 & 0.00  \\ 
  21-22 & 0.92\sym{***} & 0.92 & 0.00   \\ 
  23 & 0.84\sym{***} & 0.84 \sym{***}& 0.00   \\ 
  24 & 0.93\sym{***} & 0.93 \sym{***}& 0.00  \\ 
  25 & 0.93 \sym{***}& 0.92 \sym{***}& 0.01 \\ 
  26 & 0.91 \sym{***}& 0.92 \sym{***}& -0.01 \\ 
  27-28 & 0.93\sym{***} & 0.95 \sym{***}& -0.02 \\ 
  29 & 0.95 \sym{***}& 0.93 \sym{***}& 0.02 \\ 
  30-33 & 0.93 \sym{***}& 0.91\sym{***} & 0.02 \\ 
  34-35 & 0.92 \sym{***}& 0.89 \sym{***}& 0.02 \\ 
  36-37 & 0.91 \sym{***}& 0.88 \sym{***}& 0.03 \\ 
  45 & 0.76 \sym{***}& 0.77 \sym{***}& -0.01 \\ 
  50-52 & 0.94 \sym{***}& 0.92 \sym{***}& 0.03 \\ 
  55 & 0.88 \sym{***}& 0.83 \sym{***}& 0.05 \\ 
  60-64 & 0.88 \sym{***}& 0.83 \sym{***}& 0.05  \\ 
  65-67 & 0.88 \sym{***}& 0.84 \sym{***}& 0.04  \\ 
  70-74 & 0.82 \sym{***}& 0.80 \sym{***}& 0.03  \\ 
  75-95 & 0.87 \sym{***}& 0.85 \sym{***}& 0.03 \\ 
  AVG & 0.90 & 0.88 &   \\ 
   \bottomrule
   \multicolumn{4}{l}{\footnotesize \sym{*} \(p<0.05\), \sym{**} \(p<0.01\), \sym{***} \(p<0.001\)}\\
   \multicolumn{4}{l}{ISIC Rev. 3.1 Code} \\
   \multicolumn{4}{l}{Structural RCA based on domestic valued added exports} \\
   \multicolumn{4}{l}{Benchmark Industry ISIC Rev. 3 01-05 Agriculture}\\
   \multicolumn{4}{l}{Benchmark Country Rest of the World}\\
\end{tabular}
\end{table}
\end{comment}
\subsection*{Correlation Results \& Conclusion}
\begin{itemize}
\item The structural RCA ranking is not significantly changed if computed with domestic value-added exports.
   \item This is in contrast to \textcite{Koopman}, who found significant differences when computing the traditional BI with domestic value-added exports. 
 \item  The results may be interpreted that the production technologies in the compared  sectors are such, that sector specific sourcing and input structures do not vary strongly across sector
 \end{itemize}
 \subsection{Robustness of Correlations}
 In this section,  I report robustness checks  to assert the robustness of the correlation analysis results to changes of the normalization and the sample coverage. \par
In the table \cref{tab:productivity_gross_va_without_row} the Pearson and Spearman correlation coefficients indicate a strong degree of similarity. Overall the correlations of the sample without Rest of the World show a perfect correlation or nearly perfect correlation between gross exports and value added exports. The correlations coefficients are slightly higher for the Spearman and the Pearson correlation without Rest of the World compared to the full sample. The results suggest that the very strong correlation between the RCA obtained with gross exports and domestic value-added exports is, therefore, robust to changes in the reference country.  
\begin{table}[H]
\footnotesize
\centering
  \resizebox{\textwidth}{!}{\begin{minipage}{\textwidth}
\caption{Correlations of structural RCA based on gross exports \& domestic value-added exports - Robustness to changes in time , normalization and sample coverages}
\label{tab:productivity_gross_va_without_row}
\begin{tabular}{l*{8}{c}}
  \toprule
 &\multicolumn{2}{c}{with Rest of the world 2005}& \multicolumn{2}{c}{with Rest of the world 1995} & \multicolumn{2}{c}{without Rest of the World} & \multicolumn{2}{c}{estimation sample*} \\ 
 ISIC & Pearson & Spearman & Pearson & Spearman & Pearson  & Spearman & Pearson  & Spearman \\ 
  \midrule
15-16 & 0.99 & 0.98 & 1.00 & 0.99 & 0.99 & 0.98 & 1.00 & 1.00 \\ 
  17-19 & 0.99 & 0.99 & 1.00 & 0.99 & 0.99 & 0.99 & 1.00 & 0.99 \\ 
  20 & 1.00 & 1.00 & 1.00 & 1.00 & 1.00 & 1.00 & 1.00 & 1.00 \\ 
  21-22 & 1.00 & 0.99 & 1.00 & 0.99 & 1.00 & 0.99 & 1.00 & 1.00 \\ 
  23 & 0.98 & 0.95 & 0.98 & 0.98 & 0.98 & 0.95 & 0.98 & 0.94 \\ 
  24 & 1.00 & 0.99 & 1.00 & 0.99 & 1.00 & 0.99 & 1.00 & 0.99 \\ 
  25 & 1.00 & 0.99 & 1.00 & 0.99 & 1.00 & 0.99 & 1.00 & 0.99 \\ 
  26 & 1.00 & 1.00 & 1.00 & 1.00 & 1.00 & 0.99 & 1.00 & 0.99 \\ 
  27-28 & 0.99 & 0.99 & 1.00 & 0.99 & 0.99 & 0.99 & 1.00 & 0.97 \\ 
  29 & 1.00 & 0.99 & 1.00 & 1.00 & 1.00 & 0.99 & 1.00 & 1.00 \\ 
  30-33 & 0.99 & 0.99 & 1.00 & 0.99 & 0.99 & 0.99 & 0.99 & 0.98 \\ 
  34-35 & 1.00 & 0.99 & 1.00 & 0.99 & 1.00 & 0.99 & 1.00 & 0.97 \\ 
  36-37 & 1.00 & 0.99 & 1.00 & 1.00 & 1.00 & 0.99 & 0.99 & 0.98 \\ 
  45 & 1.00 & 0.99 & 1.00 & 0.99 & 1.00 & 0.99 & 1.00 & 1.00 \\ 
  50-52 & 1.00 & 1.00 & 1.00 & 1.00 & 1.00 & 1.00 & 1.00 & 1.00 \\ 
  55 & 1.00 & 0.99 & 1.00 & 1.00 & 1.00 & 0.99 & 1.00 & 1.00 \\ 
  60-64 & 1.00 & 0.99 & 1.00 & 1.00 & 1.00 & 0.99 & 1.00 & 0.99 \\ 
  65-67 & 1.00 & 1.00 & 1.00 & 1.00 & 1.00 & 1.00 & 1.00 & 1.00 \\ 
  70-74 & 1.00 & 1.00 & 1.00 & 1.00 & 1.00 & 1.00 & 1.00 & 1.00 \\ 
  75-95 & 1.00 & 1.00 & 1.00 & 1.00 & 1.00 & 0.99 & 1.00 & 0.99 \\ \midrule
  AVG & 1.00 & 0.99 & 1.00 & 0.99 & 1.00 & 0.99 & 1.00 & 0.99 \\ 
   \bottomrule
   \multicolumn{5}{l}{Benchmark Industry 01-05 Agriculture}\\
\multicolumn{5}{l}{Benchmark Country Rest of the World \&  USA}\\
 \multicolumn{5}{l}{*The estimation sample covers the same countries as the} \\
 \multicolumn{5}{l}{estimation sample for $\theta$}
  \end{tabular}
      \end{minipage}}
\end{table}
The correlation reported in the sample above show nearly perfect or perfect correlation of the structural RCA for both samples. The robustness check confirms the previous result, that the structural RCA ranking is unchanged when value-added exports .    %Interpretation
 The correlations between the measures in the estimation sample show nearly perfect or perfect correlation. The robustness check confirms the previous result, that the RCA and the implied ranking are not changed by using value-added exports .
Overall I conclude that the conclusions from the correlation results are robust to the changes of the country coverage and changes in the normalization.
