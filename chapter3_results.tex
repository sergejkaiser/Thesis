%!TEX root=/Users/sergej/Documents/Master/Thesis/main.tex 
\section{Empirical results structural RCA}
\label{sec:cross-section}
In this section, I describe the empirical results of the two step procedure to estimate the structural RCA measure.  The first step, is to estimate $\theta$ with OLS and IV methods. %Further, to assess the robustness of the estimates I further present the PPML results.  
In the second step I obtain the structural RCA indicator form a fixed effects regression. I discuss the association of the RCA indicator for the three indicators in two steps. First, I will compare the structural RCA ranking for the country pair China and the USA in all sectors. Second, I will present the association of structural RCA against GDP plot, which will show whether the RCA ranking is stronger for high income countries.
\par 
In the table  1 I show the cross-sectional results for the year 2005 of OLS and IV estimation. In the IV estimations, I instrumented the regressor productivity with research and development expenditures as in  \textcite{costinot}. First, I note that in both tables as theoretically predicted the point estimates of $\theta$ are positive and significant.  In the columns 2-5 I report the IV estimates of $\theta$  for different samples regarding  the industry coverage. and country coverage to show that estimates are robust. In column (4) I reduce the sample countries to include only high income countries based on the world bank classification for 2005. 
%\par
%Endogenous trade policy might affect our estimate as follows. First, the econometric error term in the estimation equation 2.1 has a structural interpretation as variable trade cost. Following the outline of \textcite{costinot2010} suppose that a country engages in a endogeneous trade policy which raises the trade barriers based on the pentration of imports. In terms of our regression this would lead to a correlation of our dependent variable gross exports $\tilde{x}^k_{i,j}$ and the error term $\epsilon_{i,j}^k$. The substation of the relation between policy induced trade cost and gross exports back into the estimation equation introduces a negative term, which includes the productivity variable, as it was shown analytically in \textcite{costinot2010}. The bias term would introduce a bias of $\theta$ towards zero. Since the EU is a free trade zone a sample only including this countries, is a possibility to control for trade policy effects.
% I report the estimates of the full sample excluding mining and agriculture and the sample excluding only mining.
 %\par 
% The OLS estimates for gross exports shows a small yet statistically strongly significant coefficient . The IV estimates in the columns 2-5 are in the range of 15.29 -- 17.64 and are all statistically strongly significant. The estimates show a decreasing patter between column 2-4. It follows  from the interpretation of $\theta$ as the inverse of industry productivity differences that as the parameter is smaller in column (3) than in column (2)  that the sample excluding agriculture and mining shows a higher cost dispersion compared to the full sample. The cost dispersion in the service and manufacturing industries in the sample is higher than the agriculture and mining industry. Moreover, since the estimate of $\theta$ is smaller in column (4) than in column (3) I conclude that agriculture shows a higher cost dispersion than mining. In addition, comparing column (5)with column (1) shows that endogenous trade protection does not significantly bias the estimate of $\theta$, as the estimate in column (5) is in the 95\% interval of column (1). The differences in $\theta$ between both tables in the same columns are not significant. 
\par 
The OLS estimates for gross exports and backward value-added exports shows a small yet statistically strongly significant coefficient. The IV estimates in the columns (2)-(4) are significantly increased with an estimated $\theta$ between 12.63 and 14.68. I interpret the increase of the IV estimate as an indicator that the independent variable is endogenous, since otherwise both estimates should show no significant difference \textcite{hausman1978}. 
From a substantial point of view, there are mainly two reasons, why I use an instrument to account for potential endogeneity of $\tilde{z}^k_i $. First, the estimates of $\theta$ might be biased because of measurement error in the international price data.  The bias of an measurement error would cause the estimate to be biased towards zero \parencite{AngristKrueger01}. 
%If an explanatory variable is measured with additive
%random errors, then the coefficient on that variable in a bivariate ordinary least
%squares regression will be biased toward zero in a large sample. The higher the
%proportion of variability that is due to errors, the greater the bias. Given an
%instrument that is uncorrelated with the measurement error and the equation error
%(that is, the equation error from the model with the correctly measured data), but
%correlated with the correctly measured variable, instrumental variables provide a
%consistent estimate even in the presence of measurement error.  
Further, another reason is that a simultaneity bias may occur due to agglomeration effects \parencite{costinot}. Agglomeration effects describe  that firms locate geographically close to each other and learn about exports opportunities. This creates a link from higher export levels to  increased the productivity.  \par
The IV regression requires that the instrument is valid, which means that it satisfies the exclusion restriction and that it is relevant.  The exclusion restriction requires that the instrument R\& D has no explanatory power for exports except through productivity. The restriction is plausible  \textcite{costinot} showed that for their sample the estimates of $\theta$ were not sensitive, to changes of productivity as total factor productivity. The interpretation is that the variations of the prices, which are explained by R \& D are orthogonal to factor endowment trade motifs.
\par   The results of the first stage regression address two concerns about the validity of the IV regression, first the relevance of the instrument and whether the instrument affects the endogenous regressor in the hypothesized way. First, concerning the relevance of the instrument, the table (see appendix) shows the F-statistic of the excluded instrument in the first stage is across the specifications very high. This implies  that the instrument is highly relevant.  Further, the first stage shows a statistical significant positive effect of  R\&D on the inverse of producer prices. Concluding, the first stage confirms the expected positive effect of R\& D on the inverse of producer prices and confirms the statistical validity of the IV regression.   %Further, to gauge the relevance I comparing the magnitude of the effect of R\& D on the inverse of producer prices to  the results of the authors \textcite{costinot}, I find that the effect of R\&D on producer prices is 42 percent lower in my sample \footnote{The estimated effect of R\& D on the inverse of producer prices is 0.038 with an standard error of 0.003 in the sample of \textcite{costinot} . The estimate in my full sample is only 0.022 with an SE of 0.002. } . Since the IV estimator for one instrument may be obtained by the reduced form coefficient scaled by the first stage estimate, the reduced coefficient of the first stage likely accounts partly for the increased estimate of $\theta$ in direct comparison. I conclude that statistically the estimates are not weakly identified and diagnostic plots of the residuals against the fitted values does not suggest misspecification. However, the magnitude of the estimates is compared to the directly obtainable   
 \par
The IV estimates of $\theta$ across the samples and the dependent variables gross exports and backward value-added exports show following results. First, the comparison between the full and the sample excluding primary industries shows that the estimates are similar for $\theta$ for the dependent variable backward value-added and gross exports.  In general the estimates of $\theta$ for both dependent variables are very close and the difference is statistically indistinguishable. Second, the pattern for both dependent variables shows that the estimate base on the sample excluding the primary industries is decreased, however statistically not significant decrease compared to the full sample. Third,  the sample excluding no high-income countries and primary industries shows an statistically significantly increased estimate \footnote{ I performed an significance based  on the t-test. The distribution of test statistic is a t-distribution with $v$ degrees of freedom, where $v=(m-1)*(1+ ((1+M{-1})*B/ \bar{U}){-1})^2$ and $\bar{U}$ denotes the average within-imputation variance and $B$ denotes the between imputation variation of the estimated parameter \textcite(p.77){Rubin1987}}.  The magnitude of the increase of $\theta$ is about  28 \%. A higher estimate of $\theta$ implies a decreased dispersion of relative cost, this is what I expected for the sample with high income countries. The estimates of $\theta$ for forward value-added exports do not show a clear pattern.  \par  Comparing the IV results to the estimates to the results of \textcite{costinot}, I obtain a similar estimate to the authors results for gross exports as dependent variable ($\theta$11.1 SE 0.981). However, the authors favorite estimate uses openness corrected exports, for which they obtained an  estimate of 6.58.  They motivated using openness corrected gross exports to account for trade selection \footnote{ Trade selection denotes that a country does not produce certain goods for which they receive a low productivity draw and instead import them (\cite{costinot}).}  downward biases  the differences in observed productivity compared to the fundamental productivity. Therefore, they reasoned that the estimates of  $\theta$ with gross exports are upward biased. \par For two reasons I decided to use gross exports and value added exports without correcting for openness. First the data on the import penetration ratio is only available for the manufacturing industries, which would reduce the sample considerably. Second, I was unable to obtain a similar correction for VAX \footnote{ A possible definition openness for value-added exports might be the ratio of re-imported value added from domestic industries to VAX . However, this measure was not bounded between 0 and 1 and the results of estimating $\theta$ with this correction showed  .} 

\begin{table}[H]
\caption{Cross-section Results OLS and IV}
\scriptsize
\begin{savenotes}
\begin{subtable}{0.9\textwidth}
\centering \caption{Cross-section results I}
\label{tab:EXGR}
{
\def\sym#1{\ifmmode^{#1}\else\(^{#1}\)\fi}
\begin{tabular}{l*{4}{S}}
\toprule
            &\multicolumn{1}{c}{(1)}&\multicolumn{1}{c}{(2)}&\multicolumn{1}{c}{(3)}&\multicolumn{1}{c}{(4)}  \\ %& %\multicolumn{1}{c}{(5)}  
            &\multicolumn{1}{c}{OLS}&\multicolumn{1}{c}{Full Sample}&\multicolumn{1}{c}{Without primary}& \multicolumn{1}{c}{Without primary} \\
        & \multicolumn{1}{c}{}    & &\multicolumn{1}{c}{industries} & \multicolumn{1}{c}{industries high \footnote{\footnotesize denotes highly developed countries, in the sample this includes the following countries AUS, AUT  BEL, CAN\\
CZE, DEU, ESP, EST, FIN, FRA, GBR, GRC, HUN, IRL, ITA, JPN, KOR, LUX, NLD, POL, PRT, RUS, SVK, SVN, USA} } %\\&\multicolumn{1}{c}{EU} 
            \\ \midrule
                        \multicolumn{5}{c}{Dependent variable log gross exports in 2005} \\
\midrule
Log productivity&     0.43368 &     12.653 & 11.424 & 14.689  \\
& (0.067) & (1.331) & (1.422) & (2.13) \\

% 0.43\sym{***}&        15.97296\sym{***}&       16.66\sym{***}&     19.13\sym{***} \\%&       17.19\sym{***}\\
%            &      (0.07)         &      (.3127033)         &      (1.99)         &       (2.52)         &      
       %\\%     (2.19)         \\
\midrule

Exporter-Importer Fixed Effects & \multicolumn{1}{c}{YES}&\multicolumn{1}{c}{YES}&
\multicolumn{1}{c}{YES}%&\multicolumn{1}{c}{YES}
 &\multicolumn{1}{c}{YES}\\
 Importer-Industry Fixed Effects& \multicolumn{1}{c}{YES}&\multicolumn{1}{c}{YES}&
 \multicolumn{1}{c}{YES}&\multicolumn{1}{c}{YES} \\%&\multicolumn{1}{c}{YES}\\
\midrule
Observations     &        \multicolumn{1}{c}{18143 }        &     \multicolumn{1}{c}{18143  }        &      \multicolumn{1}{c}{16582}         &      \multicolumn{1}{c}{14449}        \\% &       \multicolumn{1}{c}{9678 }         \\
 R-squared* & 0.771& 0.197 & 0.321 & 0.141\\  
First-stage F-statistic exc. instrument    &        &  151.41&     125.60      &      85.24                \\\bottomrule% & 
   %  68.54         \\\bottomrule
\multicolumn{5}{l}{\footnotesize Heteroscedasticity robust standard errors in parentheses}\\
\multicolumn{5}{l}{\footnotesize Log Productivity is instrumented in columns 2-6 with log of R\&D expenditures}\\
\multicolumn{5}{l}{\footnotesize Without primary industries excludes the industries mining and agriculture } \\
\multicolumn{5}{l}{*  based on Fisher's z transformation}\\
\end{tabular}
}
\end{subtable}
\begin{subtable}{0.9\textwidth}
\scriptsize
\centering \caption{Cross-section results II}
\label{tab:EXGR-DVA}
{
\def\sym#1{\ifmmode^{#1}\else\(^{#1}\)\fi}
\begin{tabular}{l*{4}{S}} \toprule
      &\multicolumn{1}{c}{(1)}&\multicolumn{1}{c}{(2)}&\multicolumn{1}{c}{(3)}&\multicolumn{1}{c}{(4)}  \\
      %&\multicolumn{1}{c}{(5)} 
                      &\multicolumn{1}{c}{OLS}&\multicolumn{1}{c}{Full Sample}&\multicolumn{1}{c}{Without primary}& \multicolumn{1}{c}{Without primary} \\
        &   &  & \multicolumn{1}{c}{industries} & \multicolumn{1}{c}{industries high}\\ \midrule
                               \multicolumn{5}{c}{Dependent variable log backward value-added exports in 2005} \\ \midrule
 Log Productivity&   0.476 & 12.911 & 11.762 & 15.080 \\ 
   &( 0.066) & (1.340) & (1.447) & (2.180) \\ 
%            &      (0.07)         &      (1.78)         &      (2.01)         &      (2.54)                
%           \\%    (2.17)         \\
           
Exporter            Importer Fixed Effects & \multicolumn{1}{c}{YES}&\multicolumn{1}{c}{YES}&\multicolumn{1}{c}{YES}&\multicolumn{1}{c}{YES}% &\multicolumn{1}{c}{YES}
\\
 Importer Industry Fixed Effects & \multicolumn{1}{c}{YES}&\multicolumn{1}{c}{YES}&\multicolumn{1}{c}{YES}&
\multicolumn{1}{c}{YES} \\ \midrule %&\multicolumn{1}{c}{YES}\\\midrule
Observations     &        \multicolumn{1}{c}{18085}         &       \multicolumn{1}{c}{  18085 }         &     \multicolumn{1}{c}{  16538 }        &      \multicolumn{1}{c}{  14412 }      \\
%  &        \multicolumn{1}{c}{  9659 }      \\
  R-squared* &  0.775 & 0.180 & 0.304 & 0.128 \\ 
%F-statistic second stage &         50.68        &       54.56        &       60.05        &   60.82            
   % \\%  63.94        \\ 
%Kleibergen-Paap rk Wald F-statistic&                      &  109.81         &       85.94         &     68.84              
First-stage F-statistic of exc. instrument &  &151.41&     125.60      &      85.24           \\
    \bottomrule %     70.78      \\ 
\multicolumn{5}{l}{\footnotesize Heteroscedasticity robust standard errors in parentheses}\\
\multicolumn{5}{l}{\footnotesize Log Productivity is instrumented in columns 2-6 with log of R\&D expenditures}\\
\multicolumn{5}{l}{\footnotesize Without primary industries excludes mining and agriculture industry} \\
\multicolumn{5}{l}{*  based on Fisher's z transformation}\\
\end{tabular}
}
\end{subtable}
\begin{subtable}{0.9\textwidth}
\scriptsize
\centering \caption{Cross-section results III}
\label{tab:EXGR-FDDVA}
{
\def\sym#1{\ifmmode^{#1}\else\(^{#1}\)\fi}
\begin{tabular}{l*{4}{S}} \toprule
      &\multicolumn{1}{c}{(1)}&\multicolumn{1}{c}{(2)}&\multicolumn{1}{c}{(3)}&\multicolumn{1}{c}{(4)}
        \\%&\multicolumn{1}{c}{(5)} 
                    &\multicolumn{1}{c}{OLS}&\multicolumn{1}{c}{Full Sample}&\multicolumn{1}{c}{Without primary}& \multicolumn{1}{c}{Without primary} \\
        &   &  & \multicolumn{1}{c}{industries} & \multicolumn{1}{c}{industries high}\\ \midrule
                                    \multicolumn{5}{c}{Dependent variable log forward value-added exports in 2005} \\ \midrule
           %  \\ \midrule
 Log Productivity&          -0.01908         &  9.286 & 10.325 & 10.218  \\%   12.87\sym{***}&       16.10\sym{***}&       16.10\sym{***} \\%&       12.73\sym{***}\\
            &      (0.0454)        &(0.868) & (1.291) & (1.199) \\ %     (1.15)         &      (2.05)         &      (1.94)         \\%&      (1.39)         \\
Exporter   Importer Fixed Effects & \multicolumn{1}{c}{YES}&\multicolumn{1}{c}{YES}&\multicolumn{1}{c}{YES}&
\multicolumn{1}{c}{YES} \\ %&\multicolumn{1}{c}{YES}\\
 Importer Industry Fixed Effects & \multicolumn{1}{c}{YES}&\multicolumn{1}{c}{YES}&\multicolumn{1}{c}{YES} &\multicolumn{1}{c}{YES} 
 \\ \midrule
Observations     &        \multicolumn{1}{c}{16727}     &        \multicolumn{1}{c}{16727}         &        \multicolumn{1}{c}{15271}         &        \multicolumn{1}{c}{14095}       \\ %/         %\multicolumn{1}{c}{8918}         \\
R-squared* &        0.882 & 0.475 & 0.431 & 0.488     \\  %&       83.89         \\
First-Stage F-statistic of exc. instrument&                       &     151.41&     125.60      &      85.24           \\     %&       96.01         \\
\bottomrule
\multicolumn{5}{l}{\footnotesize Heteroscedasticity robust standard errors in parentheses}\\
\multicolumn{5}{l}{\footnotesize Log Productivity is instrumented in columns 2-6 with log of R\&D expenditures}\\
\multicolumn{5}{l}{\footnotesize Without primary industries excludes the industries mining and agriculture} \\
\multicolumn{5}{l}{*  based on Fisher's z transformation}\\
\end{tabular}
}
\end{subtable}
\end{savenotes}
\end{table}
\endinput