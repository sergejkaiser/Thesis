%!TEX root=/Users/sergej/Documents/Master/Thesis/main.tex 
\chapter{Introduction}
\label{cha:Intro} 
%Establish the relevance of the thesis - three steps 1. Increasing product fragmentation - extend  - discussed in policy 2. This leads to increased double counting - hence necessary to use VA measures 
In the last four decades, international trade has become increasingly characterized by international production fragmentation (IPF). \textcite{feenstra98} characterized IPF as the break down the vertically-integrated production process.   Moreover, \textcite{baldwin2014}  described IPF as an increase of complex trade flows of labor, human capital, and investment between countries.  \textcite{Johnson2012}  studied the evolution of IPF over four decades. They find that IPF started to increase in the 1970s, and stagnated in the following decade, and it accelerated very strongly since the 1990s. The increase of IPF by threefold since the 1990s compared to the pre-1990s highlights this acceleration according to the authors.
%International product fragmentation leads to more trade of intermediate outputs and international supply chains \textcite{baldwin2014} 
Further,  \textcite{timmer_gvc} documented the extent of IPF  by comparing the foreign content of goods in 1995 and 2005  for 560 products. They find that for 86 \% of the products the foreign content increased. Moreover, Baldwin and Lopez-Gonzalez (2014) showed for the same time span that the final goods share of exports declined for all 16 manufacturing sectors, which they interpret it as a sign of increased IPF. 
\par %  IFPleads to double counting and important changes comparative advantage %IFP unreliable Gross measures
A consequence of IPF is that measures of international trade like gross exports including an increasing share of double counting, due to repeated border crossings of goods in the production process \parencite{feenstra98}. Further, the authors argued that IPF is fostering the trade in intermediate goods, which worsens the problem of double counting.  Also, \textcite{johnson}  emphasized that gross measures are unreliable indicators of the domestic share of
value-added in exports and about the origins of value-added embodied in final goods.  Baldwin and Lopez-Gonzalez (2014)  noted that traditional measures as gross exports of international trade do not give an accurate description of IPF. The criticism addressed about gross exports initiated a new literature on the correct measure of the value of exports. \par 
In this literature several authors (\cite{johnson},\cite{daudin2011}, \cite{Koopman}) argued to focus on value-added exports instead of gross exports. An important motivation is that as \textcite{daudin2011} noted, value-added exports allow to answer the policy question correctly \textquotedbl{}Who produces for whom\textquotedbl{} with international trade statistics. Recently in this literature \textcite{Koopman} provided an accurate
accounting framework to decompose gross exports into value-added exports and pure double counting, which refined previous frameworks.% a bit short
 \par
%% It is important changes comparative advantage
%Further, the impact of IPF on comparative advantage has been studied  (\cite{Arndt_K},  \cite{bhagwati98}, Baldwin and Lopez-Gonzalez 2014, \cite{Koopman}). First,  \parencite{bhagwati98} hypothesized that firms are increasing 'footloose' and locate in countries along the  'kaleidoscope of comparative advantage,' therefore small changes in cost are changing the pattern and as a consequence Ricardian comparative advantage has become more volatile.  Similar,  \textcite{Arndt_K} hypothesized that IPF  increased the potential for specialization of countries according to their Ricardian comparative advantage. Further, Baldwin and Lopez-Gonzalez (2014)  stated that international IPF revolutionized global manufacturing and results in transformative changes %, which they term as \textquotedbl{}Globalization 2 unbundling\textquotedbl{},  
%which lead to the denationalization of comparative advantage. A renewed empirical interest in Ricardian comparative sources of trade patterns, as \textcite{Leromain} noted, was initiated by the studies \textcite{eaton} and \textcite{costinot}. Both studies highlighted the importance of technological comparative advantage on trade patterns.
% Hence, this literature highlights to reexamine comparative advantage in the light of IPF. 
%According to the authors in global production networks, parts, and components of final goods are produced increasingly international with firm-specific know-how on management and production crossing borders. This leads the authors to conclude that the   \textquotedbl{}Globalization 2 unbundling\textquotedbl{} 
 \par % Motivate usage of VA data to look at RCA
Hence the motivation for this thesis to study the impact of IPF on technological sources of comparative advantage. Related to the thesis, \textcite{Koopman} analyzed the effects of international product fragmentation on RCA rankings with  value-added
exports. The authors concluded that the RCA ranking changed significantly when calculated with value-added exports.  In this thesis, I contribute to the literature studying the impact of IPF on RCA  by computing the structural RCA measure of  \textcite{costinot} for both gross exports and value-added exports.
\par % BI is empirical and theoretical not the right indicator; the costinot measure is favorable for theoretical and empirical reasons.
A limitation of the RCA ranking of  \textcite{Koopman} is their use of the \textcite{Balassa} index (BI) for the RCA ranking. The literature on the BI showed that the index has both empirical and theoretical limitations.
 First, from a theoretical perspective,\textcite{Leromain} critize that the BI is based on observed trade flows whereas comparative advantage in the Ricardo model is based on the fundamental productivity of countries before trade occurs \parencite{Leromain}.  Second, the empirical analysis of \textcite{yeats} concluded that the BI has poor ordinal ranking qualities. Third, the statistical analysis of \textcite{hinloopen2001} found that the distribution of the BI shifted notably between countries. Therefore, the authors concluded
that cross-country comparisons are problematic.  In addition, \textcite{Leromain} noted in their analysis that the distribution of the BI has poor time stability. In contrast, 
  %Right Measure and Right Indicator
\textcite{Leromain} found favorable empirical properties of the structural RCA from \textcite{costinot}. Especially, their results showed that the distribution of the new RCA measure is symmetric, has good ordinal ranking qualities and is stable over time.  Therefore the structural RCA of \textcite{costinot} is empirically and theoretical reasons the better indicator to analyze technological comparative advantage.    \par
The analysis conducted in this thesis showed that domestic value-added exports do not substantially alter structural RCA or the implied RCA ranking. The simple and rank correlation coefficients comparing the rankings showed a very strong correlation and the coefficients were close to one. Further, I find that my results are robust to changes in the country coverages and the base year. The results are in contrast to the finding of \textcite{Koopman}. \par
% Connection to Network Analysis 
 A second objective of the thesis is to link the structural RCA to relative network centrality. The hypothesis is motivated by the literature on shock propagation in networks. In this literature \textcite{acemoglu2012} formulated a reduced form model of interactions among many economic actors in a network.  In their setup of one production factor (labor) with industries using a Cobb-Douglas production technology, stochastic productivity shocks and consumers with one labor unit and  Cobb-Douglas preferences they showed that network centrality is the first-order characteristic of an industry in one layer production network. In this model network centrality describes how much an industry contributes to a production network concerning \$. Similar, the stochastic interpretation of trade shares states that trade shares reflect the pattern of cost advantages. The objective of this thesis is, therefore, to empirically analyze the association between relative network centrality and structural RCA. \par
I find that both measures are strongly correlated and that the result is robust to changes in the country coverages and the base year. Moreover, I find that the rank correlation is higher than the simple correlation implying that the relation between the measures is rather monotonic instead of linear. 
 \par % redo this
In what follows, I will decribe the assumptions  in the model \textcite{costinot}, which are necessary to understand the derivation of  the structural  RCA indicator. Further, I describe the steps necessary to construct the sample. Further, I will sketch the interpretation and the construction of the measure of value-added exports.  Further, in chapter two I present the results of the estimations and the correlation of structural RCA with gross exports and value added exports. In chapter three I describe the definition of the international trade network and define network centrality. Moreover, I present the empirical results of comparing relative network centrality and structural RCA. In chapter four I conclude. 
\endinput 
 %First I review important findings of studying IPF with NA and then I turn to
%nternational production fragmentation has been extensively studied with the tools of Network analysis (NA).
% First \textcite{NA_prod} has found that the assortativity of industries in the World Input-Output table of the WIOD has increased in the international trade network. Thus, industries in countries with a similar number of trade linkages are increasingly directly trading with each other. Moreover, the authors found that assortativity was correlated with the increased share of foreign content in final demand, which they see as evidence of international product fragmentation. Moreover, \textcite{de2011world} measured the density of the International trade network, which is the ratio of the number of trade  linkages and the maximum of achievable number of trade linkages, over a time of four decades.% Further  \textcite{de2011world} found using NA that the probability of that countries in the same continent trade is higher, indicating a regionalization of value chains similar to the results of \textcite{baldwin2014}.
%The authors results suggest that the density of the world trade network increased and based on centrality they have concluded that the increase was widespread. 
%The studies summarized above have shown that NA applied to international trade can achieve valuable insights. A future direction of NA research is to study the mechanisms of international shock propagation \parencite{de2011world}. \par 
%Describe results of shock propagation in networks literature % Second motivate from this second objective to look at the correspondence of NA and RCA
%The financial crisis motivated economist studied the shock propagation in one layer production networks  (\cite{acemoglu2015}, \cite{acemoglu2012}). 

%Motivate NA for International trade - Link results from NA to IPF - Create link from NA to shock propagation Literature
% Another way to look at the international product fragmentation is to use network analysis (NA) tools. The use of NA shifts the focus of the analysis to the relations between countries, the structure of those and the systemic aspects or the network as whole \parencite{de2011world}. NA analysis of \cite{serano} has shown that the topology of the international trade network in the year 2000 shows characteristics of a complex network \footnote{A complex network, is a network where every node is connected to every other node.}. Further, the authors have analyzed the distribution of trade linkages distribution at the country level. Further, they have found that the international trade network shares a small world property, which means that the average path \footnote{the path measures the number of countries needed to connect two countries \textcite{de2010comparing}} length between countries increases logarithmically with the size of the network. Moreover, they have found a high probability that countries, which are connected to a common trade partner, are also connected. In addition, their analysis showed that countries with a similar number of import/export linkages are likely to be connected with each other.  \textcite{de2011world} noted that these results are important, as the complex network property implies that international trade is mutually beneficial. 
%Furthermore \textcite{serano} have found that the distribution of trade linkages has a scale-free property, which implies a large heterogeneity of trade linkages among countries.
%Analyzing the international trade network at the industry level \textcite{de2010comparing} found that network measures indicate a much lower density than on the aggregated level. 
%%The authors have noted that the rise of product fragmentation leads to a reversal
%of the trend of an increasing share of world income of the G7 \footnote{The G7 is shorthand of the seven leading industry nations, which are USA, GER, JPN, FRA, ITL, GRB and CAN.}. In
%the last twenty years from 2010 to 1990, the relative share of global income of the G7 nations 
%decreased by 26 \% to only 46 \%. The decline
%of the relative world share of income of the G7 nations is remarkable
%if one compares it to past values. The authors have shown that the last
%time the relative share of the global income of the G7 was at a similar level was at the beginning of the 20th century.
